

\chapter*{MỞ ĐẦU} % Tên của chương
\addcontentsline{toc}{chapter}{MỞ ĐẦU} % Thêm tên chương vào mục lục

\label{Chapter6} % Để trích dẫn chương này ở chỗ nào đó trong bài, hãy sử dụng lệnh \ref{Chapter0} 

%----------------------------------------------------------------------------------------
Thuật toán tìm kiếm đường đi trong mê cung là một trong những bài toán cơ bản và thường được sử dụng trong môn cấu trúc dữ liệu và giải thuật. Việc tìm kiếm đường đi trong mê cung là một ví dụ điển hình để giới thiệu các thuật toán tìm kiếm và các khái niệm cơ bản như đồ thị, quy hoạch động và các kỹ thuật tối ưu.

Mô phỏng thuật toán tìm kiếm đường đi trong mê cung giúp sinh viên hiểu và tự tìm hiểu các giải thuật, tạo cảm hứng và khai thác triệt để các kiến thức được học. Không chỉ dừng lại ở các mô tả trên lý thuyết, mô phỏng các thuật toán tìm kiếm đường đi trong mê cung giúp sinh viên biết cách thực hiện áp dụng vào các bài tập trên thực tế.

Với sức mạnh của công nghệ, mô phỏng thuật toán tìm kiếm đường đi trong mê cung càng trở nên dễ dàng hơn bằng việc sử dụng các công cụ và thư viện phần mềm để tạo ra một môi trường mô phỏng và thực hiện việc đánh giá các thuật toán tìm kiếm đường đi. Điều này giúp cho môn học cấu trúc dữ liệu và giải thuật có thể được học một cách trực quan và hiệu quả hơn.
